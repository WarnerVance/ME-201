\documentclass[twocolumn, letterpaper, 12pt, twoside]{article}
\usepackage{graphicx}
\usepackage{amsmath}
\usepackage{wrapfig}
\usepackage{geometry}
\usepackage{float}
\geometry{margin=1in}

  
\title{%
    Crane Project Report \\
\large ME 201
}

\author{Warner Vance, Sam Sierra, Malini Krejcarek}
\date{23 August 2024}

\begin{document}
\maketitle

\section{Introduction}
The goal of this project was to make a crane capable of lifting a heavy object as fast as possible using as little mass as possible. 
We were given a chassis, motors, pulleys,  and a microcontroller, everything else we had to make ourselves. 
This includes several aluminum scaffolding components, a base made of HDPE, structural braces, and a wooden support implement. 

\section{Design}
\subsection{Pulley System}
\label{sec:Pulley}
We decided early on make a simple crane, so that we could focus our efforts on optimization.  
We wanted to lift 2.1 kg, and we designed the crane around it. 
Our first major decision was the pulley system we planned to use. 
We did some quick math and determined that a block and tackle system that split the tension in half would be best. 
Using a spreadsheet, we came up with an equation for the relationship between torque and rotational speed of a motor. 
This equation was then multiplied by two to account for the fact that two motors were being used. 
This equation was then put into EES along with an equation that related the torque to the radius of the pulley and the weight of the object being pulled up.

\begin{figure}[H]
    \centering
    \includegraphics[width =\linewidth]{Torque_Speed.png}
    \caption{A graph of the relationship between the radius of the drive pulley and the vertical velocity of the object being lifted. Torque delivered from the motors is also included for reference.}
    \label{fig:Torque_Speed}
\end{figure}

We derived the vertical velocity of the object from the radius of the drive pulley and the rotational speed of the motor. 
The subsequent parametric study and graph showed that 1 inch was the idea drive pulley radius because it maximized the vertical velocity of the object being lifted.  This can be seen in Figure \ref{fig:Torque_Speed}.
This translated to a torque of around .28 n-m


\subsection{Counterweight}
\label{sec:Counterweight}
We then moved on to calculating both the mass and the position of the counter weight. 
First we repeated an earlier lab procedure to find the center of mass of the crane using equations    . 

\begin{align}
        w_{\text{tip}}= & m_{\text{tip}} * 9.8 \\
        t_{\text{tip}}= & w_{\text{tip}}*x_{\text{boom}}\label{eq:Torque 1} \\ 
        t_{\text{tip}}= & w_{\text{crane}}*x_{\text{cm}}\label{eq:Torque 2} \\
        x_{\text{cm}}= & \frac{t_{\text{tip}}}{w_{\text{crane}}}
\end{align}

\begin{equation}
    \begin{split}
        \text{Where:} \\
        m_{\text{tip}} & \text{ is the mass of the object that makes the crane tip} \\
        w_{\text{tip}} & \text{ is the weight of } m_{\text{tip}} \\
        x_{\text{out}} & \text{ is the x from the center of the crane to the where the mass is hung} \\
        t_{\text{tip}} & \text{ is the torque from the object that makes the crane tip} \\
        w_{\text{crane}} & \text{ is the weight of the crane without the counterweight} \\
        x_{\text{cm}} & \text{ is the x from the lifting side wheel to the center of mass of the crane.} \\
    \end{split}
\end{equation}

Equation \ref{eq:Torque 1} and \ref{eq:Torque 2} are made possible by the fact that at the moment of tipping the sum of the moments must be zero because nothing is moving.
\end{document}